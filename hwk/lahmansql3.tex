\documentclass[nobib]{tufte-handout}
\usepackage{amsmath}
\usepackage{setspace}
\usepackage{hyperref}
\usepackage{booktabs}


\setlength{\textheight}{9in} \setlength{\topmargin}{-.5in}
\setlength{\textwidth}{6.5in} \setlength{\oddsidemargin}{0in}
\setlength{\evensidemargin}{0in}

\title{COMP 310 --- SQL (Home)work 3 --- Multi-table Tomfoolery }
\author{  }
\date{ }
 

\begin{document}
\maketitle

Let's explore batting performance of all-stars vs. regular batters.  Here are a few constraints on this investigation: 
\begin{enumerate}
    \item Year - 1982
    \item Only batters that had at least 400 plate appearances (PA = AB + BB +HBP + SF + SH)
    \item Compute slugging (see homework 1) for each batter 
    \item National League only
\end{enumerate}

Our goal is two separate queries: one for all-stars and one for non-all stars. Sort by slugging (high to low).
\begin{itemize}
    \item (two columns) First and last name
    \item (two columns) team name and year (yes it's 1982, just include it)
    \item (one column) stint 
    \item (one column) plate appearances
    \item (six columns) slugging and all the stats used to compute it (H,AB,2B,3B,HR)
\end{itemize}


\end{document}
\documentclass[10pt]{article}
\usepackage{amsmath}
\usepackage{setspace}
\usepackage{hyperref}
\usepackage{booktabs}

\setlength{\textheight}{9in} \setlength{\topmargin}{-.5in}
\setlength{\textwidth}{6.5in} \setlength{\oddsidemargin}{0in}
\setlength{\evensidemargin}{0in}

\title{Syllabus \\ COMP 310 \\ Database Theory and Design}
\author{  }
\date{Fall 2024}

\begin{document}
\maketitle

\section{Logistics}
\begin{itemize}
\item \textbf{Where: }
\begin{itemize}
\item Class: Center for Science and Business (CSB), Room 303
\end{itemize}
\item \textbf{When: } MWF 10--10:50am
\item \textbf{Instructor: } Logan Mayfield
\begin{itemize}
\item \textit{Office: } Center for Science and Business (CSB), Room 344
\item \textit{Phone: } 309-457-2200 % chktex 8
\item \textit{Website: } \url{http://jlmayfield.github.io/}
\item \textit{Email: } lmayfield \textit{at} monmouthcollege \textit{dot} edu
\item \textit{Office Hours: }  M 9-10am. Tu, 10:30am - 11:30am. W, 2-3pm. Th, 10:30-11:30am. F, 2-3pm. By appointment.
\end{itemize}
\item \textbf{Website: } \url{http://jlmayfield.github.io/teaching/COMP310/}
\item \textbf{Credits: } 1 course credit
\end{itemize}
\emph{Note: This Syllabus is subject to change based on specific class needs. Significant deviations from the syllabus will be discussed in class.}

\section{Textbook}

\noindent
Christopher Painter‑Wakefield. \textit{A Practical Introduction to Databases}.  runestoneinteractive.org. \url{https://runestone.academy/ns/books/published/practical_db/index.html} % chktex 8

\section{Description and Content}

In this course we examine the design and usage of databases with an emphasis on relational databases. Databases are an integral part of a great number of software systems. In this course we first get a handle on how to approach and utilize databases in the context of software development. Once we're comfortable working with existing databases we'll turn our attention to the process by which one designs and implements the right database for a given project. Students can expect to get a lot of hands on experience with SQL including accessing a local database, a remote database, and interacting with a database through high-level programming languages such as Python.  In addition to the case studies found in the text, we'll look at applications to sabermetrics (Baseball statistics) using the well known Lahman database of baseball data. Students will also examine the design of the Lahman database from several perspectives.



\subsection{Workload}
% number of/details on midterms, finals, project, homeworks, quizes, etc

Time spent on work for this course will likely vary by student and will, in general, vary week to week. On average, this course should require about 13 hours of work per week per student.  The following table provides a rough estimate of the distribution of this time over different course components.
\begin{center}
\begin{tabular}{ll}
\underline{Assignment Type} & \underline{Time/week} \\
Lectures/Class       & 3 hours/week \\
Homework          &  2 hours/week \\
Exam Study Time    &  1 hours/week \\
Projects          &  3 hours/week \\
Reading+Unstructured Study &  2 hours/week \\
\bottomrule
 & 11  hours/week
\end{tabular}
\end{center}

Time outside the classroom will be spent on practice exercises, practice problems, projects, and self-evaluation letters. The course will also include regular exams.
\begin{center}
  \begin{tabular}{ll}
    \underline{Category} & \underline{Number of Assignments} \\
    Self-Check Exercise Sets &  15-19 \\
    Problem Sets & 4-5 \\ 
    Group Projects & 2 \\
    Exams & 4-6\\
    Portfolio Review \& Self-Evaluation Meetings & 4-5
  \end{tabular}
\end{center}

\subsection*{Self-Check Exercises and Problem Sets}

Exercises and problems from the book will serve two purposes in this course: they will reinforce ideas learned during class time or they will be used to introduce new ideas and seed collaborative learning sessions in class.  Expect self-check problem sets from each chapter of the text. They are built into the text and should be done as part of the reading. Students can expect to spend some class time either reviewing problems or working them together. Additional problems, not found in the text, will be assigned as well. 

\subsection*{Exams}

Exams are meant to test your understanding of and ability to apply ideas covered previously in the course. They are a gut check the current state of your learning. They let you answer the question, ``Do I know the material as well as I think I know the material.''  Compared to homework, they are higher stakes assessments of your learning because you lack the safety net that is your notes, the text, and other references.

For the most part, exams will be done in class and will be announced ahead of time in order for you to prepare.  Expect a mix of exams that take all or most of a class period and involve multiple questions or a multi-part problem along with shorter pop-exams that are unannounced, involve one or two quick questions, and will only take up a small portion of the start of a class period.

\subsection*{Group Projects}

We will use a hands-on, applied database project and a more broad research paper. The project will allow you to take a deeper dive into the material covered in class and engage in problems that would be too large for basic exercises.  The paper asks you to explore the world of databases beyond the scope of the class and then share those findings with your classmates. These larger assignments are what exercises, problems, and the in-class work associated with them are preparing you for.  They are the big-show and should be treated as such. A great deal of learning happens from these more open-ended, larger-scale, explorations of the class material. You'll carry out one project at the completion of the SQL portion of the text and another after the section on Database Design. Projects will include a presentation component. 

\subsection*{Portfolio Review \& Self-Evaluation}

Self-reflection and self-evaluation is a critical component of learning and vital to a growth mindset.We will keep a portfolio of the work you do throughout the semester. Much of this will be done automatically by our assignment management and version control software. At regular intervals throughout the semester you will meet, one-on-one, with me to \textit{present your porfolio}, review items from your portfolio that best gauge how well you're doing at meeting the course goals and expectations, and discuss how that success maps to a letter grade. For more details about the process visit \url{https://jlmayfield.github.io/teaching/ungrading/howto-portfolio}.



\section{Ungrading \& Final Grades}

This class is largely ungraded. That means your assignments will not be graded for points and your final grade
is not determined by a point-based, numerical grading system. You will get feedback on your work but you will
see points on nothing. You don't earn points for doing work or getting something correct nor do you lose points
for getting something wrong. We're here to learn. Doing the work is how we do that and getting things wrong
some or most of the time is part of learning. For more details visit \url{https://jlmayfield.github.io/teaching/ungrading/howto-portfolio}.

\subsection{Self-Evaluation \& Final Course Grades}

Throughout the semester you'll be asked to engage in regular self-evaluation. This process is described in
detail in additional documentation. Part of the process includes you self-assigning a course grade based on
your self-evaluation. Your self-evaluation and self-assigned grade are then discussed with me in a one-on-one
meeting during which we'll agree upon your current grade. The key here is that \textit{your self-evaluation
and self-assigned grade begins the conversation, not my assigned points.}

Below are some general rules of thumb we'll try to stick to when talking about grades. They relate grades to
course competency expectations and Monmouth College policy.
\begin{itemize}
  \item \textbf{A} - Exceeding course expectations.
  \item \textbf{B} - Meeting and occasionally exceeding course expectations.
  \item \textbf{C} - Meeting course expectations. \textit{This is the minimum grade required to continue on to COMP152. So, a C means you can be successful in a class that builds upon the things learned in this class.}
  \item \textbf{C-} - Mostly meeting course expectations. \textit{This is the minium grade that counts towards a major.}
  \item \textbf{D} - Occasionally meeting course expectations, but mostly not. \textit{Grades in the D range earn credit towards graduation but fall below GPA requirements.}
  \item \textbf{F} - Did not meet course expectations.
\end{itemize}

My hope is that the self-evaluation and self-directed grading process provides a lot of flexibility in terms
of how you can achieve success in this course and meet your grade goals. If you ever have questions or concerns
about self-evaluations and grades, then I'm more more than willing to discuss them with you at any time.

\subsubsection{Participation, Attendance, \& Timely Work}

I do not have strict attendance and deadline policies, per se, but I do have clear expectations. These
expectations are baked into the dispositional attribute of the course competencies. This attribute
includes things like being \textit{professional, responsible, responsive, and self-directed.}

As far as I'm concerned, signing up for this class means you agree to coming to class and lab,
being on time for class and lab, doing assigned work and submitting it on time, and generally participating
in all the class has to offer.  That being said, life happens and people have different priorities.
You might need to miss class or extend a deadline.  So long as you communicate with me about it, as a
professional would with a co-worker, then we won't have a problem. If you simply skip class without
warning, always show up late, or regularly fail to do assigned work in a timely manner, then I expect that
those failures to meet dispositional expectations to be reflected in your self-evaluation.

There is one exception to my ``no grade-based policy'' on assignments and deadlines and that is the
self-evaluations and reflections. The self-evaluation process is critical to this class and in no way
optional. \textbf{If you fail attend the portfolio review meetings or always show up completely un-prepared
then I reserve to give you a final grade of D or lower for the course.} You'll find I can be pretty relaxed
about a lot of other assignments and deadlines, but I draw the line at the self-evaluation process.

\subsubsection*{Academic Honesty}

You don't learn by trying to pass off someone's work as your own. In an ungraded class it makes even less sense to cheat and steal work from somewhere else.  There are no points, you gain nothing from it and you certainly will learn nothing from it. In this ungraded class, academic dishonesty is still not tolerated.

From the Monmouth College Academic Honesty Policy:
\begin{quote}
  ``We view academic dishonesty as a threat to the integrity and intellectual mission of our institution. Any breach of the academic honesty policy - either intentionally or unintentionally - will be taken seriously and may result not only in failure in the course, but in suspension or expulsion from the college. It is each student’s responsibility to read, understand and comply with the general academic honesty policy at Monmouth College, as defined here in the Scots Guide, and to the specific guidelines for each course, as elaborated on the professor’s syllabus.''

  ``The following areas are examples of violations of the academic honesty policy:
  \begin{enumerate}
  \item Cheating on tests, labs, etc;
  \item Plagiarism, i.e., using the words, ideas, writing, or work of another without giving appropriate credit;
  \item Improper collaboration between students, i.e., not doing one’s own work on outside assignments specified as group projects by the instructor;
  \item Submitting work previously submitted in another course, without previous authorization by the instructor.''
  \end{enumerate}

  ``Please note that this list is not intended to be exhaustive.''
\end{quote}

In this course, any violation of the academic honesty policy will have varying consequences depending on the severity of the infraction as judged by the instructor.  Expect violations to be reported to the appropriate Dean and to weaken your case for higher grades at the end of the course. Severe violations can result in an F for the course and expulsion from the course. Do your own work. If you even think something you're doing could be construed as academically dishonest, then ask for guidance and clarification first.


\section{Academic Support \& Accessibility}

\subsection*{Support Services}
The Academic Support and Accessibility Services Office offers free resources to assist Monmouth College students with their academic success. Programs include Supplemental Instruction for difficult classes, Drop-In and appointment tutoring, and individual Academic Coaching. Our office is here to help all students excel academically, since every student can work toward better grades, practice stronger study skills, and manage their time better. Please email academicsupport@monmouthcollege.edu for assistance.

\subsection*{Accessibility Services}
If you have a disability and/or medical/mental health condition or had academic accommodations in high school or another college, you may be eligible for academic accommodations at Monmouth College under the Americans with Disabilities Act (ADA). Monmouth College is committed to equal educational access. To discuss any of the services offered, please call or meet with Jennifer Sanberg, Associate Director of Academic Support and Accessibility Services. The ASAS office is located on the first floor of the Hewes Library, opposite Einstein’s Bros Bagel. They can be reached at 309-457-2257 or via email at: academicsupport@monmouthcollege.edu


\subsection{Calendar}

\textit{This calendar aspirational and is subject to change based on the circumstances of the course. A more detailed, regularly updated calendar can be found on the course website. }


\begin{center}
\begin{tabular}{lllll}
\underline{Week} & \underline{Dates} & \underline{Assignments Due} & \underline{Chapter(s)} & \underline{Notes} \\
1 & 8/21--8/23  &  & 1.1 & \\
2 & 8/26--8/30 & P. Set 1. & 1.2--1.3 & \\
3 & 9/2--9/6 & Exam 1. & 1.4 & NO CLASS M. - Labor Day. \\
4 & 9/9--9/13  & P. Set 2. & 1.4,1.8--1.9 &  \\
5 & 9/16--9/20 & Exam 2. & 1.10--1.11 & \\
6 & 9/23--9/27 &  & Project & \\
7 & 9/30--10/4 & Project 1. & Project. 2.1  & \\
8 & 10/7--10/9 & & 2.1--2.2 & FALL BREAK Th-F  \\
9 & 10/14--10/18 & P. Set 3. Exam 3. & 2.2--2.3 &  \\
10 & 10/21--10/25  & & 2.3--2.4 & \\
11 & 10/28--11/1 & P. Set 4. Exam 4. & 1.5--1.7. Project. & \\
12 & 11/4--11/8 &  & Project. & \\
13 & 11/11--11/15 & Project 2. & Project. 3.1 & \\
14 & 11/18--11/22 &  & 3.1--3.3 & \\
15 & 11/25--11/26 & P. Set 5. & 3.4 & THANKSGIVING W-F \\
16 & 12/2--12/6 & & 3.4 & NO CLASS F. - Reading Day Th.\\
17 & 12/9 & Exam 5. Mon. 12/9 11:30am -- 2:30 pm & &  \\
\end{tabular}
\end{center}

\end{document}

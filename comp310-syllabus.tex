\documentclass[10pt]{article}
\usepackage{amsmath}
\usepackage{setspace}
\usepackage{hyperref}
\usepackage{booktabs}

\setlength{\textheight}{9in} \setlength{\topmargin}{-.5in}
\setlength{\textwidth}{6.5in} \setlength{\oddsidemargin}{0in}
\setlength{\evensidemargin}{0in}

\title{Syllabus \\ COMP 310 \\ Database Theory and Design}
\author{  }
\date{Fall 2022}

\begin{document}
\maketitle

\section{Logistics}
\begin{itemize}
\item \textbf{Where: }
\begin{itemize}
\item Class: Center for Science and Business (CSB), Room 303
\end{itemize}
\item \textbf{When: } TTh 11--12:15am
\item \textbf{Instructor: } Logan Mayfield
\begin{itemize}
\item \textit{Office: } Center for Science and Business (CSB), Room 344
\item \textit{Phone: } 309-457-2200 % chktex 8
\item \textit{Website: } \url{http://jlmayfield.github.io/}
\item \textit{Email: } lmayfield \textit{at} monmouthcollege \textit{dot} edu
\item \textit{Office Hours: }  \textbf{By appointment.} MW 9:30 -- 11:00. TTh 9:30 -- 10:30
\end{itemize}
\item \textbf{Website: } \url{http://jlmayfield.github.io/teaching/COMP310/}
\item \textbf{Credits: } 1 course credit
\end{itemize}
\emph{Note: This Syllabus is subject to change based on specific class needs. Significant deviations from the syllabus will be discussed in class.}

\section{Textbook}

\noindent
Connolly, Thomas \& Begg, Carolyn. \textit{Database Systems: A Practical Approach to Design, Implementation, and Management}. Sixth Edition. Pearson. 2015. % chktex 8

\section{Description and Content}

In this course we examine the design and usage of databases with an emphasis on relational databases. Databases are an integral part of a great number of software systems. In this course we first get a handle on how to approach and utilize databases in the context of software development. Once we're comfortable working with existing databases we'll turn our attention to the process by which one designs and implements the right database for a given project. Students can expect to get a lot of hands on experience with SQL including accessing a local database, a remote database, and interacting with a database through high-level programming languages such as Python.  In addition to the case studies found in the text, we'll look at applications to sabermetrics (Baseball statistics) using the well known Lahman database of baseball data. Students will also examine the design of the Lahman database from several perspectives.

By the end of the course students will have covered chapters 1,2,4--7,10,12,14,16--19 of the text.  There will be a required set of review questions and exercises with each chapter. The expectation is that students will have read the chapter and begun working on the review questions and exercises prior to addressing that material in class such that class time can be dedicated to deeper discussion and active problem solving.


\subsection{Workload}
% number of/details on midterms, finals, project, homeworks, quizes, etc

Time spent on work for this course will likely vary by student and will, in general, vary week to week. On average, this course should require about 13 hours of work per week per student.  The following table provides a rough estimate of the distribution of this time over different course components.
\begin{center}
\begin{tabular}{ll}
\underline{Assignment Type} & \underline{Time/week} \\
Lectures/Class       & 4 hours/week \\
Homework          &  2 hours/week \\
Exam Study Time    &  2 hours/week \\
Projects          &  3 hours/week \\
Reading+Unstructured Study &  2 hours/week \\
\bottomrule
 & 13  hours/week
\end{tabular}
\end{center}

Time outside the classroom will be spent on practice exercises, practice problems, projects, and self-evaluation letters. The course will also include regular exams.
\begin{center}
  \begin{tabular}{ll}
    \underline{Category} & \underline{Number of Assignments} \\
    Question and Exercise Sets &  11 \\
    Group Project & 1 \\
    Paper \& Presentation &  1 \\
    Exams & 4-6\\
    Self-Evaluation Letters & 4 \\
  \end{tabular}
\end{center}

\subsection*{Exercises and Problems}

Exercises and problems from the book will serve two purposes in this course: they will reinforce ideas learned during class time or they will be used to introduce new ideas and seed collaborative learning sessions in class.  Expect problem sets from each chapter of the text.  They will typically be assigned as we begin a chapter and collected shortly after finishing the chapter. Students can expect to spend some class time either reviewing problems or working them together. When exercises are used to introduce new material, they will often be due at the next class period and you will be expected to have good-faith attempts ready to share and discuss.

\subsection*{Exams}

Exams are meant to test your understanding of and ability to apply ideas covered previously in the course. They are a gut check the current state of your learning. They let you answer the question, ``Do I know the material as well as I think I know the material.''  Compared to homework, they are higher stakes assessments of your learning because you lack the safety net that is your notes, the text, and other references.

For the most part, exams will be done in class and will be announced ahead of time in order for you to prepare.  Expect a mix of exams that take all or most of a class period and involve multiple questions or a multi-part problem along with shorter pop-exams that are unannounced, involve one or two quick questions, and will only take up a small portion of the start of a class period.

\subsection*{Projects & Papers}

We will use a hands-on, applied database project and a more broad research paper. The project will allow you to take a deeper dive into the material covered in class and engage in problems that would be too large for basic exercises.  The paper asks you to explore the world of databases beyond the scope of the class and then share those findings with your classmates. These larger assignments are what exercises, problems, and the in-class work associated with them are preparing you for.  They are the big-show and should be treated as such. A great deal of learning happens from these more open-ended, larger-scale, explorations of the class material. You can expect the project to take place close to midterm and the paper to be due at the end of the semester.

\subsection*{Self-Evaluation Letters}

Self-reflection and self-evaluation is a critical component of learning and vital to a growth mindset. You'll write four self-evaluation letters to me through out the course of the semester. In these letters you'll evaluate the state of your learning, present evidence of successes, examples of ongoing challenges, and address how well you believe you're meeting course and personal goals and expectations. As you'll read below, these letters and the conversations we have as a result of the letters will, by and large, determine your grade in the course.

Letters will be one to two pages in length and will be submitted at the start of class, after the first project, at midterm, and at the end of the semester along with the final project.


\section{Ungrading \& Final Grades}

This class is ungraded. That means your assignments will not be graded and your final grade is not determined by a point-based, numerical grading system. You will get feedback on your work but you will see points on nothing. You don't earn points for doing work or getting something correct nor do you lose points for getting something wrong. We're here to learn. Doing the work is how we do that and getting things wrong some or most of the time is part of learning. If I could do away with assigning a final grade I would, but the college requires it.

\subsection{Self-Evaluation \& Final Course Grades}

Throughout the semester you'll be asked to engage in regular self-evaluation, the purpose of which is to evaluate your learning and begin a dialog about your progress with the instructor. The result of the evaluation will be a letter to the professor. This process is also where and how your final grade will be determined. As a part of your self-evaluation you'll give yourself a grade and discuss why you think that grade is appropriate given what you presented in your self-evaluation of learning. You start the grading conversation. After having read your self-evaluation, I will discuss with you the proposed grade and we'll come to an agreement on your current grade.

As a class and in our post-self-evaluation dialogues, we'll be discussing some standards and expectations for determining final grades in the absence of a point-based system. The cornerstones of these expectations are the things we need to do in order to learn and grow: meaningful, regular, and intensional work in order to meet established learning goals and standards.


\subsubsection{Doing Good Work: Participation, Attendance, \& Timely Work}

You are expected to be in class and in lab. Missing class and lab is like skipping practice for a sport, for a play, for a musical performance, or for anything else in which people expect to prepare ahead of time for an event. Learning in this class is meant to be a social exercise. We will explore new ideas as a class, ask and answer questions of one another, and generally use each member of the class as a resource. If people don't show up, then it not only robs them of the chance to grow through the day's activities but it also robs the members of the class of the chance to benefit from their unique perspective as an individual. That being said, missed classes are often unavoidable. The key is missing for a good, largely unavoidable reasons and trying to keep the absences to a minimum.

Assignments will be given, collected, and returned when it best benefits your learning. By doing an assignment late you risk getting out of sync with the class and not getting as much out of the assignment as you otherwise could. It can also make you an unequal contributor to group work as you will not have practiced and reinforced the same material as your classmates. Still, things happen and we can't always meet deadlines.  What's important is that you make all reasonable efforts to get all assignments done and in on time.


\subsubsection*{Academic Honesty}

You don't learn by trying to pass off someone's work as your own. In an ungraded class it makes even less sense to cheat and steal work from somewhere else.  There are no points, you gain nothing from it and you certainly will learn nothing from it. In this ungraded class, academic dishonesty is still not tolerated.

From the Monmouth College Academic Honesty Policy:
\begin{quote}
  ``We view academic dishonesty as a threat to the integrity and intellectual mission of our institution. Any breach of the academic honesty policy - either intentionally or unintentionally - will be taken seriously and may result not only in failure in the course, but in suspension or expulsion from the college. It is each student’s responsibility to read, understand and comply with the general academic honesty policy at Monmouth College, as defined here in the Scots Guide, and to the specific guidelines for each course, as elaborated on the professor’s syllabus.''

  ``The following areas are examples of violations of the academic honesty policy:
  \begin{enumerate}
  \item Cheating on tests, labs, etc;
  \item Plagiarism, i.e., using the words, ideas, writing, or work of another without giving appropriate credit;
  \item Improper collaboration between students, i.e., not doing one’s own work on outside assignments specified as group projects by the instructor;
  \item Submitting work previously submitted in another course, without previous authorization by the instructor.''
  \end{enumerate}

  ``Please note that this list is not intended to be exhaustive.''
\end{quote}

The complete Monmouth College Academic Honesty Policy can be found on the College web page by clicking on ``Student Life'' then on ``Scot’s Guide'' in the navigation bar to the left, then ``Academic Regulations'' in the navigation bar at the left.  Or you can visit the web page directly by typing in this URL: \url{https://ou.monmouthcollege.edu/life/residence-life/scots-guide/academic-regulations.aspx}

In this course, any violation of the academic honesty policy will have varying consequences depending on the severity of the infraction as judged by the instructor.  Expect violations to be reported to the appropriate Dean and to weaken your case for higher grades at the end of the course. Severe violations can result in an F for the course and expulsion from the course.

\subsubsection*{Evidence of Learning: Course Competencies}

This course has a set of competencies that combine areas of knowledge, levels of skill, professional dispositions with a specific task. The result is a broad set of outcomes for the class and the benchmarks by which you can and will assess your learning and overall success in this course. As you progress through the course, you should be able to point to specific assignments, parts of assignments, or things you did in service of completing an assignment as evidence that you are meeting or exceeding the course competencies.


\section{Academic Support \& Accessibility}

\subsection*{Support Services}
The Academic Support and Accessibility Services Office offers free resources to assist Monmouth College students with their academic success. Programs include Supplemental Instruction for difficult classes, Drop-In and appointment tutoring, and individual Academic Coaching. Our office is here to help all students excel academically, since every student can work toward better grades, practice stronger study skills, and manage their time better. Please email academicsupport@monmouthcollege.edu for assistance.

\subsection*{Accessibility Services}
If you have a disability and/or medical/mental health condition or had academic accommodations in high school or another college, you may be eligible for academic accommodations at Monmouth College under the Americans with Disabilities Act (ADA). Monmouth College is committed to equal educational access. To discuss any of the services offered, please call or meet with Jennifer Sanberg, Associate Director of Academic Support and Accessibility Services. The ASAS office is located on the first floor of the Hewes Library, opposite Einstein’s Bros Bagel. They can be reached at 309-457-2257 or via email at: academicsupport@monmouthcollege.edu


\subsection{Calendar}

\textit{This calendar aspirational and is subject to change based on the circumstances of the course. A more detailed, regularly updated calendar can be found on the course website. }


\begin{center}
\begin{tabular}{llll}
\underline{Week} & \underline{Dates} & \underline{Assignments Due} & \underline{Chapter(s)}\\
1 & 8/24 --- 8/26  &  & 1 \\
2 & 8/29 --- 9/2 & Self Eval. 1 & 1,2\\
3 & 9/5 --- 9/9 & Hwk 1  &  4,6 \\
4 & 9/12 --- 9/16  & Hwk 2 & 6 \\
5 & 9/19 --- 9/23 & Exam. Hwk 3. & 6,7 \\
6 & 9/26 --- 9/30 & Hwk 4. Self Eval. 2 (NO CLASS THURSDAY) & 7 \\
7 & 10/3 --- 10/7 & Hwk 5  & 5 \\
8 & 10/10 --- 10/14 & Project. FALL BREAK (Th-F)   &   \\
9 & 10/17 --- 10/21  &  & 10,11  \\
10 & 10/24 --- 10/28  & Hwk 6. Self Eval. 3.  & 11,12    \\
11 & 10/31 --- 11/4 & Hwk 7.  & 12,14 \\
12 & 11/7 --- 11/11 &  Hwk 8. Exam. & 14  \\
13 & 11/14 --- 11/18 & Hwk 9.  &  16,17 \\
14 & 11/21 --- 11/25 & Exam. (THANKSGIVING W-F)  &  17 \\
15 & 11/28 --- 12/2 & Hwk 10.  & 18 \\
16 & 12/5 --- 12/10 & Papers \& Presentations. (READING DAY. Th)  &  \\
17 & 12/12 --- 12/14 & (Exam Time: 12/13 11:30am - 2:30pm) Exam. Self Eval. 4. &
\end{tabular}
\end{center}

\end{document}

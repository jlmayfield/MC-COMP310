\documentclass[10pt]{article}
\usepackage{amsmath}
\usepackage{setspace}
\usepackage{hyperref}
\usepackage{booktabs}

\setlength{\textheight}{9in} \setlength{\topmargin}{-.5in}
\setlength{\textwidth}{6.5in} \setlength{\oddsidemargin}{0in}
\setlength{\evensidemargin}{0in}

\title{Syllabus \\ COMP 310 \\ Database Theory and Design}
\author{  }
\date{Fall 2018}

\begin{document}
\maketitle

\section{Logistics}
\begin{itemize}
\item \textbf{Where: }
\begin{itemize}
\item Class: Center for Science and Business (CSB), Room 303
\end{itemize}
\item \textbf{When: } MTWF 10--10:50am
\item \textbf{Instructor: } James \textit{Logan} Mayfield
\begin{itemize}
\item \textit{Office: } Center for Science and Business (CSB), Room 344
\item \textit{Phone: } 309-457-2200 % chktex 8
\item \textit{Website: } \url{http://jlmayfield.github.io/}
\item \textit{Email: } lmayfield \textit{at} monmouthcollege \textit{dot} edu
\item \textit{Office Hours: }  1:30--2:30 WF. 9:30--11:00Th. By Appointment.
\end{itemize}
\item \textbf{Website: } \url{http://jlmayfield.github.io/teaching/COMP310/}
\item \textbf{Credits: } 1 course credit
\end{itemize}
\emph{Note: This Syllabus is subject to change based on specific class needs. Significant deviations from the syllabus will be discussed in class.}

\section{Textbook}

\noindent
Connolly, Thomas \& Begg, Carolyn. \textit{Database Systems: A Practical Approach to Design, Implementation, and Management}. Sixth Edition. Pearson. 2015. % chktex 8

\section{Description and Content}

In this course we examine the design and usage of databases with an emphasis on relational databases. Databases are an integral part of a great number of software systems. In this course we first get a handle on how to approach and utilize databases in the context of software development. Once we're comfortable working with existing databases we'll turn our attention to the process by which one designs and implements the right database for a given project. Students can expect to get a lot of hands on experience with SQL including accessing a local database, a remote database, and interacting with a database through high-level programming languages such as Python.  In addition to the case studies found in the text, we'll look at applications to sabermetrics (Baseball statistics) using the well known Lahman database of baseball data. Students will also examine the design of the Lahman database from several perspectives.

By the end of the course students will have covered chapters 1,2,4--7,10,12,14,16--19 of the text.  There will be a required set of review questions and exercises with each chapter. The expectation is that students will have read the chapter and begun working on the review questions and exercises prior to addressing that material in class such that class time can be dedicated to deeper discussion and active problem solving.


\section{Expectations and Policies}

Students are expected to carry themselves in a mature and professional manner in this course. Towards this end, there are a few classroom policies by which every student is expected to abide.
\begin{itemize}

\item \textit{Late Assignments: } In general, late assignments will \textit{not} be accepted.  Students who feel they have a justified reason for submitting an assignment late may set up an appointment to meet with the instructor and plead their case.  Students are more likely to get extensions on assignments when they are asked for in advance rather than the day the assignment is due.

\item \textit{Attendance: } \textbf{Repeated absences and late arrivals to class will quickly reduce a student's participation grade to zero.}  The occasional late arrival or missed class is one thing, but being habitually late and regularly missing classes is disruptive and not fair to your classmates.

\item \textit{Participation: }  Cellphone and computer usage in class for non-class related activities is strongly discouraged.  All devices should be set to silent when in class.  If a student's usage of technology becomes a distraction to their classmates or the instructor, then that student's participation grade will suffer.  If the instructor or a classmate has to inform a student that they're being a distraction, then their use of technology has already gone too far.  When in doubt, err on the side of caution.

\end{itemize}


\subsection{Collaboration}

In general, students are encouraged to make use of the resources available to them.  This means it is OK to seek help from a friend, the tutor, the instructor, the internet, etc.  However, \textit{copying of answers and any act worthy of the label of ``cheating'' or ``plagiarism'' is never permissible!. Students should always be able to reproduce an answer on their own, and if they cannot then they likely \textbf{do not really known the material.}} All of the Monmouth College rules on academic dishonesty apply.  A student found in violation of the rules should be prepared to face the consequences of their actions. If a student needs help understanding the rules, then please seek out the instructor before doing something that might violate academic honesty policies.

\section{Grades}

This courses uses a standard grading scale.  Assignments and final grades will not be curved except in rare cases when its deemed necessary by the instructor.  Percentage grades translate to letter grades as follows:

\begin{center}
\begin{small}
\begin{tabular}{lcl}
\underline{Score} & & \underline{Grade} \\
94--100 & & A \\
90--93 & & A- \\
88--89 & & B+ \\
82--87 & & B \\
80--81 & & B- \\
78--79 & & C+ \\
72--77 & & C \\
70--71 & & C- \\
68--69 & & D+ \\
62--67 & & D \\
60--61 & & D- \\
0--59 & & F
\end{tabular}
\end{small}
\end{center}


You are always welcome to challenge a grade that you feel is unfair or calculated incorrectly.  Mistakes made in your favor will never be corrected to lower your grade.  Mistakes made not in your favor will be corrected.  \textit{Basically, after the initial grading, your score can only go up as the result of a challenge.}

\subsection{Workload}
% number of/details on midterms, finals, project, homeworks, quizes, etc

The course workload is as follows:
\begin{center}
  \begin{tabular}{ll}
    \underline{Category} & \underline{Number of Assignments} \\
    Question and Exercise Sets &  11 \\
    Project/Paper &  2 \\
    Exams & 7\\
  \end{tabular}
\end{center}

There will be no dedicated midterm or final exam. There are just exams.  Exams will generally focus on material covered since the previous exam.

\subsection{Grade Weights}

Your final grade is based on a weighted average of particular assignment categories.  You should be able to estimate your current grade based on your scores and these weights.  You may always visit the instructor \textit{outside of class time} to discuss your current standing.

\begin{center}
  \begin{tabular}{ll}
  \underline{Category} & \underline{Weight} \\
    Exams & 49 \% \\ %7% each
    Project/Paper & 24 \% \\ %12% each
    Homework & 22 \% \\ % 2% each
    Participation &  5 \%
  \end{tabular}
\end{center}


\subsection{Course Engagement Expectations}

The weekly workload for this course will vary by student but on average should be about 13 hours per week.  The follow tables provides a rough estimate of the distribution of this time over different course components for a 16 week semester.
\begin{center}
\begin{tabular}{ll}
\underline{Assignment Type} & \underline{Time/week} \\
Lectures/Class       & 4 hours/week \\
Homework          &  2 hours/week \\
Exam Study Time    &  2 hours/week \\
Projects          &  3 hours/week \\
Reading+Unstructured Study &  2 hours/week \\
\bottomrule
 & 13  hours/week
\end{tabular}
\end{center}


\subsection{Calendar}

\textit{This calendar is subject to change based on the circumstances of the course.}

\begin{center}
\begin{tabular}{llll}
\underline{Week} & \underline{Dates} & \underline{Assignments Due} & \underline{Chapter(s)}\\
1 & 8/22 --- 8/24 &  & 1,2 \\
2 & 8/27 --- 8/31 & Hwk 1. & 2,4  \\
3 & 9/3 --- 9/7 & Hwk 2. &  4,6 \\
4 & 9/10 --- 9/14 & Exam 1 (M). &  6 \\
5 & 9/17 --- 9/21 & Hwk 3. & 7 \\
6 & 9/24 --- 9/28 & Hwk 4. Exam 2 (F). & 7  \\
7 & 10/1 --- 10/5 & Project 1. &  5 \\
8 & 10/8 --- 10/10 & Hwk 5. Exam 3. FALL BREAK(ThF) &  \\
9 & 10/16 --- 10/19 & FALL BREAK(M) Hwk 6. & 10 \\
10 & 10/22 --- 10/26 &  & 12 \\
11 & 10/29 --- 11/2 & Hwk 7. Exam 4 (F). & 14  \\
12 & 11/5 --- 11/9 & Hwk 8.  &  16 \\
13 & 11/12 --- 11/16 & Exam 5 (M). Hwk 9. Project 2. & 16,17  \\
14 & 11/19 --- 11/20 & Exam 6 (Tu).THANKSGIVING(WThF) & 17 \\
15 & 11/26 --- 11/30 & Hwk 10.  & 17 \\
16 & 12/3 --- 12/5 & Hwk 11.  READING DAY (Th) & \\
Final's Week & 12/10 (8:00am --- 11:00am) Exam 7. & \\
\end{tabular}
\end{center}

\end{document}
